\chapter{\label{cha:intro}Introduction}

This chapter introduces all aspects that form the context and
motivations of the graduate project. We will describe \ldots

\section{Terminology}

The term \emph{reverse engineering} is defined by Chikofsky and Cross as
``\emph{The process of analyzing a subject system with two goals in mind:
(1) to identify the system's components and their interrelationships; and,
(2) to create representations of the system in another form or at a higher
level of abstraction}'' \cite{CC90}.

Note that this Terminology section is not a required section but
merely an example.

\section{Research Question(s)}

The research questions investigated in this thesis are:

\begin{enumerate}[label=\textbf{RQ\arabic*}]
\item How should I use this thesis template style?
\item How can I import this thesis style into ShareLatex?
\item Any other question you can think of?
\end{enumerate}

\lipsum{10} % add some pseudo content
